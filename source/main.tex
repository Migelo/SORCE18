\section{IAU2018}

%\frame
%{
%%...................................................................................................
%\note<1>[]{}
%%...................................................................................................
%	\frametitle{}
%	\begin{itemize}
%		\item 
%	\end{itemize}
%	\centering
%%	\includegraphics[width=115mm]{images/}
%}

\frame[t]
{
%...................................................................................................
\note<1>[]{TSI varies on timescales from minutes to decades. So what is the main contributor to this variability?}
%...................................................................................................
	\frametitle{\alert{T}otal \alert{S}olar \alert{I}rradiance}
	\begin{itemize}
		\item TSI -- spectrally integrated solar radiative flux at 1 AU from the sun \\
	\end{itemize}

	\centering
	\includegraphics[width=130mm]{images/tsi}
	
}


\frame
{
%...................................................................................................
\note<1>[]{If we look at the solar spectra we see it is full of spectral lines...}
%...................................................................................................
	\frametitle{Spectra of the individual components}
	\centering
	\LARGE Solar spectra \\
	\includegraphics[width=130mm]{images/solar_spectra}
	
}



\frame[t]
{
%...................................................................................................
\note<1>[]{Then we can do simulations w/ and w/o lines and get the this result...so we can see the lines really are the main contributor to the TSI variability}
%...................................................................................................
	\frametitle{Importance of lines for variability}
	\begin{itemize}
		\item TSI -- \alert{T}otal \alert{S}olar \alert{I}rradiance, i.e. integrated over wavelengths
		\item SSI -- \alert{S}pectral \alert{S}olar \alert{I}rradiance, depends on wavelength 
		\item $\triangle$ -- difference between the solar minima and maxima
		%\item The increase of the TSI at maximum of the activity cycle compared with minimum is directly attributed to the variability in spectral lines
	\end{itemize}
	\centering
	\includegraphics[width=90mm]{images/continuum}
}


\frame[t]
{
%...................................................................................................
\note<1>[]{}
%...................................................................................................
	\frametitle{Importance of lines for variability}
	\vspace{2.1em}
	\centering
	\includegraphics[width=.73\textwidth]{images/timescales_2}
}


\frame[t]
{
%...................................................................................................
\note<1>[]{...careful and appropriate treatment of spectral lines is necessary for TSI variability calculations.}
%...................................................................................................
	\frametitle{Importance of lines for variability}
	\begin{itemize}
		\item \alert{25\%} of the variability comes from molecular lines $\rightarrow$ accurate linelists are required
	\end{itemize}
	\centering
	\includegraphics[width=.73\textwidth]{images/timescales_3}
}

\frame
{
%...................................................................................................
\note<1>[]{If we want to obtain the emergent spectra from a 3D MHD cube which gives us the structure of the magnetic fields we pierce it with 1k rays and then calculate the emergent spectra from those 1D atmospheres. There are multiple methods how to speedup RT and here is were my work comes in. I will show you my work on ODF and the improvements we have made on them. }
%...................................................................................................
	\frametitle{1.5D simulations}
	\centering
	\includegraphics[width=150mm]{images/1_5D}
}



%%...................................................................................................
%\note<1>[item]{The Str\"omgren filters B and Y, as well as the Kepler passband or the COROT filter, are broadly used for observations.
%
%Therefore, these filters are of particular interest when radiative transfer calculations are performed. Currently, this is computationally demanding due to a huge amount of atomic and molecular lines.
%
%In this talk, I will show that a SIGNIFICANT speed up can be achieved by using optimized opacity detribution functions, which we call OODF!
%
%Let’s start with the example of the Strömgren Y filter. As you can see the intensity is dominated by lines….}

\frame
{
%...................................................................................................
\note<1>[]{}
%...................................................................................................
	\frametitle{Generating ODFs}
	\begin{itemize}
    \item Start with high resolution opacity	
	\end{itemize}

	\centering
	\includegraphics[width=130mm]{images/odf_generation_process_0}
}

\frame
{
%...................................................................................................
\note<1>[]{}
%...................................................................................................
	\frametitle{Generating ODFs}
	\begin{itemize}
    \item Sort wavelength points by corresponding values of opacity; monotonically increasing opacity
    \item Integral is preserved by sorting
	\end{itemize}		
	\centering
	\includegraphics[width=123mm]{images/odf_generation_process_1}
}

\frame
{
%...................................................................................................
\note<1>[item]{Take notes here.}
%...................................................................................................
	\frametitle{Generating ODFs}
	\begin{itemize}
		\item All wavelength information within the bin is lost
\end{itemize}		

		\centering
	\includegraphics[width=130mm]{images/odf_generation_process_2}
}

\frame
{
%...................................................................................................
\note<1>[]{}
%...................................................................................................
	\frametitle{Generating ODFs - Example with 10 uniform sub bins}
	\begin{itemize}
		\item Approximate the sorted opacity with a step-wise function
	\end{itemize}
	\centering
	\includegraphics[width=130mm]{images/odf_generation_process_3}
}
\frame
{
%...................................................................................................
\note<1>[]{}
%...................................................................................................
	\frametitle{ODF generation process}
	\begin{itemize}
	\item Mean is skewed by extreme values
	\end{itemize}
	\centering
	\includegraphics[width=130mm]{images/odf_generation_process_4}
}

\frame
{
%...................................................................................................
\note<1>[]{}
%...................................................................................................
	\frametitle{ODF performance analysis}
	\begin{itemize}
	    \item Synthesize spectra using ODFs from 1000-9000\si{\angstrom} in 10\si{\angstrom} bins
	    \item Compare the fluxes from the ODF spectrum with the high resolution spectrum in the bins
	\end{itemize}
	\centering
	\includegraphics[width=120mm]{images/odf_vs_non_odf}
}


\frame[t]
{
%...................................................................................................
\note<1>[]{}
%...................................................................................................
	\frametitle{Possible solutions}
	Number of calculations goes as: \large{$n_{bins} \times n_{sub bins}$ }
	\vspace{1.6em}
	
	\centering
	
\begin{tikzpicture}[sibling distance=25em,
  every node/.style = {shape=rectangle, rounded corners,
    draw, align=center, color=white!20}]]
  \node {\Large{what to do?}}
    child { node {increase bin size \\
    				  $\rightarrow$ decrease $n_{bins}$} }
    child { node {decrease the number\\ of sub bins per bin} };
\end{tikzpicture}

%	\includegraphics[width=115mm]{images/}
}


\frame
{
%...................................................................................................
\note<1>[item]{}
%...................................................................................................
	\frametitle{Analysis of different ODFs}
	\begin{itemize}
	\item Uniform ODFs
	\end{itemize}
	
	\centering
	\includegraphics[width=130mm]{images/6_10_vs_best_4_0}
}
\frame
{
%...................................................................................................
\note<1>[]{}
%...................................................................................................
	\frametitle{Analysis of different ODFs}
	\begin{itemize}
		\item Nonuniform ODFs
		\item The last sub bin is crucial after 5000\si{\angstrom}
    \end{itemize}	  
	\centering
	\includegraphics[width=120mm]{images/6_10_vs_best_4_1}
}
\frame
{
%...................................................................................................
\note<1>[]{}
%...................................................................................................
	\centering	
	\frametitle{Best sub bin combinations using 4 sub bins}
	\includegraphics[width=140mm]{images/best_combination_finder_0}
}

\frame
{
%...................................................................................................
\note<1>[]{}
%...................................................................................................
	\centering		
	\frametitle{Best sub bin combinations using 10 sub bins}
	\includegraphics[width=130mm]{images/best_combination_finder_10_1}
}


\frame
{
%...................................................................................................
\note<1>[]{}
%...................................................................................................
	\frametitle{Best combination of 4 sub bins for Str\"omgren b}
	\begin{itemize}
		\item Total line contribution $\sim$15\%
	\end{itemize}
	\centering
	\includegraphics[width=115mm]{images/optimal_stroemgren_1_c_b}
}


\frame
{
%...................................................................................................
\note<1>[]{}
%...................................................................................................
	\frametitle{Speedups in the case of Str\"omgren \textit{b}}
	\begin{itemize}
		\item Interval length: $\sim$400\si{\angstrom}\\[20pt]
	\end{itemize}
	
	\centering
%	High resolution calculation 80 points per \si{\angstrom} $\sim$ 80 000 \\
%	ODF calculations 12 points per 10\si{\angstrom} $\sim$ 1200 \\
%	OODF calculations 3 points per 1000\si{\angstrom} $\sim$ 3 \\

\begin{tikzpicture}[sibling distance=25em,
  every node/.style = {shape=rectangle, rounded corners,
    draw, align=center, color=white!20}]]
  \node {\large{High resolution: 80 points per \si{\angstrom} $\sim$ 32 000 points}\\}
    child { node {ODF: 12 points per 10\si{\angstrom} $\sim$ 480 points \\
    \alert{\large{speedup 67 times}}} }
    child { node {OODF: 3 points for the whole bin\si{\angstrom} $\sim$ 3 points \\
    \alert{\large{speedup $\sim$11 000 times}}} };
\end{tikzpicture}

}

\frame
{
%...................................................................................................
\note<1>[]{}
%...................................................................................................
	\frametitle{Conclusions}
	\large{
	\begin{itemize}
	%\item An efficient  procedure for radiative transfer is timely for new generation of solar and stellar variability models.\\[15pt]
	\item We developed a novel method for fast spectral synthesis. \\[15pt]
	\item Found optimal sub bins for different wavelength regimes.\\[15pt]
    \item Can be tailored for different filters: Strömgren \textit{b} + \textit{y}, Kepler, PLATO  and others.\\[15pt]
    \item Significant speed up relative to standard  methods by a factor of at least two orders of magnitude.\\[10pt]
	\end{itemize}
	
	}
	\pause
		\centering \alert{\Large{Thank you for your attention!}} \\
		\includegraphics[width=25mm]{images/qr}
}


\frame
{
%...................................................................................................
\note<1>[]{}
%...................................................................................................
	\frametitle{Best combinations of 4 sub bins for Str\"omgren \textit{b}}
	\centering
	\includegraphics[width=115mm]{images/optimal_stroemgren_0_c_b}
}

\frame
{
%...................................................................................................
\note<1>[]{}
%...................................................................................................
	\frametitle{Formula: value weighted by the derivative}
	\centering
	\includegraphics[width=115mm]{images/smart_sub_bins}
}
\frame
{
%...................................................................................................
\note<1>[]{}
%...................................................................................................
	\frametitle{Ascending vs descending sort}
	\centering
	\includegraphics[width=115mm]{images/ascending_descending}
}
